\documentclass[10pt,a4paper,openright]{IEEEtran}
%\documentclass[conference]{IEEEtrans}
%%%%%%%%%%%%%%%%%%%%%%%%%%%%PACKAGES%%%%%%%%%%%%%%%%%%%%%%%%%
\usepackage{listings}
\usepackage{xcolor}
%%%%%%%%%%%%%%%%%%%%%%%Define colors%%%%%%%%%%%%%%%%%%
\usepackage{tcolorbox}
\definecolor{heading}{RGB}{34, 69, 33}
\definecolor{subheading}{RGB}{34, 69, 33}
\definecolor{subsubheading}{RGB}{34, 69, 33}
\definecolor{link}{RGB}{34, 69, 33}
\definecolor{cite}{RGB}{34, 69, 33}
\definecolor{url}{RGB}{34, 69, 33}
%\usepackage{algorithm}
\usepackage[framemethod=tikz]{mdframed}
\usepackage{times}
\usepackage{algorithmic}
\usepackage[ruled,vlined]{algorithm2e}
%%%%%%%%%%%%%%%%%%%%%%%%%%%%%%%%%%%%%%%%%%%%%%%%%%%%%%
\usepackage{sectsty}
\chapterfont{\color{heading}}  % sets colour of chapters
\sectionfont{\color{cyan}}  % sets colour of sections
\subsectionfont{\color{violet}}
\subsubsectionfont{\color{green}}
\usepackage[hmargin=0.7in,vmargin=0.7in]{geometry}
\usepackage{cite}
\usepackage{graphicx}\graphicspath{{Image/}}
\usepackage[T1]{fontenc}
\usepackage{mathptmx}
\usepackage{amsmath,amssymb,amsfonts,url,amsthm}
\usepackage{setspace}
\usepackage{subcaption}
%\usepackage[sort&compress]{natbib}
\usepackage[hidelinks]{hyperref}
\hypersetup{colorlinks,linkcolor={cyan},citecolor={red},urlcolor={blue}}
\usepackage{cleveref}
\usepackage{fancyhdr}\pagestyle{fancy}
\fancyhf{}

\usepackage{cuted}
\fancyhead[LO]{\it{\nouppercase{\rightmark}}}
%\fancyhead[LE,RO]{\thepage}
\fancyfoot{}
\usepackage[font=small,labelfont=bf]{caption}
\usepackage{emptypage}
%Defining chapter style

%chapter style ends
\usepackage[nottoc]{tocbibind}
\renewcommand{\lstlistingname}{Code}

\setcounter{secnumdepth}{3}
\setcounter{tocdepth}{3}

%%%%%%%%%%%%%%%%%%%%%%%%%%%%%%%%DEFINE STYLE %%%%%%%%%%%%%%%%%
\lstdefinestyle{chstyle}{%
	backgroundcolor=\color{gray!12},
	basicstyle=\ttfamily\small,
	commentstyle=\color{red!30},
	keywordstyle=\color{magenta},
	stringstyle=\color{blue!50!red},
	showstringspaces=false,
	captionpos=b,
	numbers=left,
	numberstyle=\footnotesize\color{gray},
	numbersep=10pt,
	stepnumber=1,
	tabsize = 1,
	frame=l,
	framerule=1pt,
	rulecolor=\color{red};
}
%%% Define Section
\mdfdefinestyle{question}{
	innertopmargin=1.2\baselineskip,
	innerbottommargin=0.8\baselineskip,
	roundcorner=5pt,
	nobreak,
	singleextra={%
		\draw(P-|O)node[xshift=1em,anchor=west,fill=yellow,draw,rounded corners=5pt]{%
			Note  };
	},
}


\newcounter{Question} % Stores the current question number that gets iterated with each new question

% Define a custom environment for numbered questions
\newenvironment{question}[1][\unskip]{
	\bigskip
	\stepcounter{Question}
	\newcommand{\questionTitle}{~#1}
	\begin{mdframed}[style=question]
	}{
	\end{mdframed}
	\medskip
}
\mdfdefinestyle{theorm}{
	innertopmargin=1.2\baselineskip,
	innerbottommargin=0.8\baselineskip,
	roundcorner=5pt,
	nobreak,
	singleextra={%
		\draw(P-|O)node[xshift=1em,anchor=west,fill=red,draw,rounded corners=5pt]{%
			Theorm  };
	},
}


\newcounter{theorm} % Stores the current question number that gets iterated with each new question

% Define a custom environment for numbered questions
\newenvironment{theorm}[1][\unskip]{
	\bigskip
	\stepcounter{theorm}
	\newcommand{\questionTitle}{~#1}
	\begin{mdframed}[style=theorm]
	}{
	\end{mdframed}
	\medskip
}
%%

\usepackage{eso-pic}
\AddToShipoutPictureBG*{%
	\AtPageUpperLeft{%
		\hspace{20pt}
		\raisebox{-140pt}{\includegraphics[scale=0.30]{logo}}
		
	}
}
%%%%%%%%%%%%%%%%%%%%%%Tilte and Author %%%%%%%%%%%%%%%%%%%%%%
\title {\LARGE{ \textbf{  \quad  Indian Institute of Information Technology Vadodara\\ (Gandhinagar Campus) \\[5mm] \LARGE { Summer Research Internship Program } \\ [5mm] \large{Programming Skill Assessment for acedemics students} }} }

%\author{Sagar Ved Bairwa [201951131]\\ Yash Beniwal [201951175] \\ Yash Jain [201951176]\\[15mm] Project Coordinator: Dr. pramit mazumdar}

\author{ Submitted By \\[2mm]
	\textbf{
	Sagar Ved Bairwa 201951131
	\\
	Vivek Kumar Goyal 201951172
	\\
	Yash Jain 201951176
	}
	\\[2mm]
	\hrule
	
	 
	
}


%\raggedbottom


%%%%%%%%%%%%%%%%%%%Document %%%%%%%%%%%%%%%%%%%%%%%
\begin{document}
\maketitle
\begin{abstract}
	This is abstract.
\end{abstract}
%\tableofcontents
%\pagestyle{fancy}
%\pagenumbering{arabic}

%\section{Introduction}
%\section{Introduction}\label{ch:ch1}
One of the most important research directions in cryptography is the design and implementation of lightweight cryptographic algorithms. The use of low-power, resource-limited devices has exploded in the last two decades. The reduction of the state size reduces the power consumption of the cipher. So it becomes a very challanging work for the community.The design principle of these lightweight stream ciphers differs significantly from the design principle of the standard stream ciphers.

Grain is a stream cipher submitted to eSTREAM in 2004 by Martin Hell, Thomas Johansson and Willi Meier. The design of Grain v1 deals with less number of hardware required. The Cipher is very useful where gate count, power consuption and memory is very limited. 

An RFID tag is a typical example of a product where the amount of memory and power is very limited. These are microchips capable of transmitting an identifying sequence upon a request from a reader. Forging an RFID tag can have devastating consequences if the tag is used e.g. in electronic payments and hence, there is a need for cryptographic primitives implemented in these tags. Today, a hardware implementation of e.g. AES on an RFID tag is not feasible due to the large number of gates needed. Grain is a stream cipher primitive that is designed to be very easy and small to implement in hardware.

It is based on two shift registers and a nonlinear output function. The cipher has the additional
feature that the speed can be increased at the expense of extra hardware.The key size is 80 bits
and the IV size is specified to be 64 bits. The cipher is designed such that no attack faster than
exhaustive key search should be possible, hence the best attack should require a computational
complexity not significantly lower than $2^{80}$.

Grain cipher is designed on specific properties. we can understand that it is not possible to have
a design that is perfect for our all purposes i.e., processors of all words lengths, all hardware
applications, all memory constraints etc. Grain is designed to be very small in hardware, using
less number of logic gates and maintain high security. we can use Grain in general application
software, when high speed in software is required . Because of this it does make sense to compare
Grain with other cipher.

Grain provides a higher security than several other well known ciphers intended to be used in
hardware applications. Well known examples of such ciphers are E0 used in Bluetooth and A5/1
used in GSM.

In FSE 2015, Armknecht and Mikhalev however proposed the stream cipher Sprout with a Grain-like architecture, whose internal state was equal in size with its secret key and yet resistant against TMD attacks. Although Sprout had other weaknesses, it germinated a sequence of stream cipher designs like Lizard and Plantlet with short internal states. Both these designs have had cryptanalytic results reported against them. In this paper, we propose the stream cipher Atom that has an internal state of 159 bits and offers a security of 128 bits. Atom uses two key filters simultaneously to thwart certain cryptanalytic attacks that have been recently reported against keystream generators.

In addition, we found that our design is one of the smallest stream ciphers that offers this security level, and we prove in this paper that Atom resists all the attacks that have been proposed against stream ciphers so far in literature. On the face of it, Atom also builds on the basic structure of the Grain family of stream ciphers. However, we try to prove that by including the additional key filter in the architecture of Atom we can make it immune to all cryptanalytic advances proposed against stream ciphers in recent cryptographic literature.

The paper is organized as follows. Section 2 provides a detailed discription of stream ciphers. Section 3 gives the Design of Grain V1 and The design and implimention of Atom is discussed in Section 4. In Section 5, concludes the paper.	
\section{Introduction}
%\section{Introduction}\label{ch:ch1}
 Programming skill Assessment for academic students is a knowledge analysis Java application whose primary goal is to evaluate the student's performance in programming languages. Every year institutes try to teach new programming languages to their students and students learn some of these of their own. The students know a lot about programming languages but they don’t know how to evaluate their performance. It is basically a programming skill assessment that has admin-defined problem sets. When a student / new learner wants to assess their skill he/she can register for the test and get her/his assessment report. After successful registration of the student, he will be redirected to our quiz system where he can evaluate his/her programming skills. It also has some basic and some advanced features for admin access so the admin can design the quizzes, can define experimental languages, and see everyone's assessment results in comma-separated values(CSV) format. 

The project report is organized as follows. Section 2 provides a literature survey of the internship program. Section 3 shows present the investigation of the internship project. section 4 shows Results and Discussion. Section 5 concludes the analysis of the report and future work.


\section{Literature survey}
The Research starts with a basic understanding of Java programming language. To find a basic understanding of java desktop applications we took the help of Java programming’s official documentation website (\href{https://www.oracle.com/technical-resources/articles/javase/new-tech.html}{https://www.oracle.com/technical-resources/articles/javase/new-tech.html}). Then we get some MySQL database knowledge from MySQL official documentation website (\href{https://docs.oracle.com/en-us/iaas/mysql-database/doc/getting-started.html}{https://docs.oracle.com/en-us/iaas/mysql-database/doc/getting-started.html})and a few youtube channels i.e. codewithharry, programming with mosh. There we get all the necessary information about queries, database creation, table creation, data insertion to the table, delete \& update data from a table, dropping \& truncate the tables, altering a table, etc. These things help us to understand the database properly.

After it, we started our project by designing the database. In the database, we created mainly 5 tables namely AdminData, UserDemographicData, Evolutiontable, Expertise, and Question bank with proper attributes. Then we started to design of fronted part of the project there we use images, text fields, password fields, check boxes, radio buttons, ok buttons, and separators. We defined all the page sizes as 1920 x 1080 px by default except for some tiny dialog boxes.

After it, we started to connect the fronted part with our backend (database). This part becomes a little bit tedious and time-consuming but we try our best to achieve our desired product. 

\section{The Present Investigation}
We divide our project into three major parts. 
\begin{enumerate}
    \item Database creation
    \item Front-end design
    \item Back-end Connection
\end{enumerate}
Let's discuss the three of the major parts of the research project. 
\subsection{Database creation}
we have prior knowledge of database design so before starting to write queries we choose all necessary things that are required to design a database i.e. table names and corresponding attributes. For designing the database, we made the Five most important tables for our \textbf{Programming Skill Assessment for the academic students} project. Here a list of all tables.
\begin{tcolorbox}
    \begin{enumerate}
    \item Demographic Data \\<UID/Name/Dob/gender/mail/…>
    \item Expertise Data \\ <UID/programming language/level/duration/time/last used>
    \item Evaluation report \\ <UID/Prog. Lang /level\# / code \#/question \#/ recorded answer/correct answer/decision/score>
    \item Question Bank \\ <UID/Prog. Lang /level\# / code \#/question \#/op1…op4/correct op/codeTimer / question timer/ score>
    \item Admin Data \\ <UID / Password /Name>
\end{enumerate}
\end{tcolorbox}
\subsubsection{Admin data table}
This table contains all the necessary details about Admin. In our research, our domain is basically student oriented so we don't put a lot data about admin. The information we colleccted for   
\begin{tcolorbox}
create table if not exists AdminData( \\
Username varchar(10) not null,\\
Userpassword varchar(15) not null,
\\Adminname  varchar(30) not null,
\\primary key (Username)
\\);
\end{tcolorbox}
\subsubsection{Demographic Details Table}
Demographic Data: This table contain all the demographic details of a participant. We don’t takeparticipant as a real user of ou application. He / She can only be join our application at the time of self-assessment or institutional assessment. This table contains all the details visible in the following query.
\begin{tcolorbox}
     create table if not exists userdemographicdata(\\ participantid varchar(15) not null, \\participantname varchar(50) not null,\\ gender varchar(10) not null,\\ dateofbirth date not null,\\ profession varchar(20) not null,\\ phonenumber varchar(12) not null,\\ email varchar(50) not null,\\ favouritelanguage varchar(20) not null, \\highesteducation varchar(30) not null, \\ state varchar(30) not null,\\ country varchar(30) not null, \\ fieldofinterest varchar(100) not null,\\ termsandconditions varchar(50) not null,\\ score int(5),\\ primary key(participantid)\\);
\end{tcolorbox}
\subsubsection{Expertise Data Table}
When a participant completes his registration, he / she will be redirected to the pre-self assessment for the programming languages. There participant need to fill all the details for the table. After successful pre-self assessment He will be redirected to a new frame where our quiz is loaded. Here is the MySQL Query for creation the Expertise Details tables.
\begin{tcolorbox}
create table if not exists expertise( \\
participantid varchar(10) not null, \\
programminglanguage varchar(15) not null,\\
languagelevel varchar(15) not null,\\
durationofuse varchar(30) not null,\\
exacttime varchar(30) not null,\\
lastused varchar(30) not null,\\
primary key(participantid)\\
);
\end{tcolorbox}

\section{Conclusion and Future Work}
We have learned many concepts through the usage of data
structures and logical thinking of how to understand the
question and respond to it through the agents present in the
environment.
\section*{Acknowlegement}
We would like to thank our instructor Mr Pramit mazumdar for
providing us with this opportunity to work on these concept
oriented problems, by solving these we have learned alot
concepts in the process.
%\appendix	
%\chapter{Software used to produce this pdf}
Build using : Tex studio [2.12.22 ]

\chapter{Software used to produce this pdf}
Build using : Tex studio [2.12.22 ]	

%\singlespacing
\section{REFERENCES} \\
Artificial Intelligence: a Modern Approach, Russell and
Norvig (Fourth edition) Chapter 1, 2, 3 2] \\
https://www.sciencedirect.com/science/article/abs/pii/0167819
189901063 \\
https://www.mygreatlearning.com/blog/best-first-search-bfs/ \\
https://www.geeksforgeeks.org/a-search-algorithm/ \\
https://en.wikipedia.org/wiki/A*searchalgorithm \\

\bibliographystyle{IEEEtran}	

\end{document}
