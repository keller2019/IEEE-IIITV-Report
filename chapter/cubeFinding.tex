\section{Cube Finding Algoritham For Grain V1}
Cube Finding Algoritham is an algorithm based on careful selection of Cube
variables which can test the non-randomness of a cipher (or a Boolean function).
The concept of Cube Finding Algoritham was first introduced by Aumasson et al. \cite{10.1007/978-3-642-03317-9_1} in 2009.
The main idea behind designing a Cube Finding Algoritham on a cipher (or a Boolean function)
is to select the Cube variables in such a way that if there is any non-randomness
in the cipher (or a Boolean function) the non-randomness must be reflected in
the corresponding superpoly. By using this kind of testing procedure one can check
several properties of a function such as the presence of any monomial in the function,
presence of neutral variables in the function, upper bound of the degree of the
function, and balancedness of the function, etc. 

\subsection{Cube Attack}
Throughout this paper, all polynomials have coefficients in GF(2).\\ let IV = $(v_1,v_2,...,v_m)$ and K = $(x_1,x_2,...,x_n)$ , $F_i(K,IV) = Y_i$ denote the $i^{th}$ output bit, if U =  $\{v_{i_1},v_{i_2},...,v_{i_k}\} \subset \{v_1,v_2,...,v_m\}$ has been chosen then $i^{th}$ output bit can been represented as 
\begin{multline}
F_i(x_1,x_2,...,x_n,v_1,v_2,...,v_m) = v_{i_1},v_{i_2},...,v_{i_k}\\P(x_1,x_2,...,x_n,V)+Q(x_1,x_2,...,x_n,v_1,v_2,...,v_m)
\end{multline}
\begin{question}
	V = $\{ v_1,v_2,...,v_m \} / U $, $P ( \cdot )$ is linear polynomial in $\{x_1,x_2,...,x_n \} \bigcup  V$ , and does not contain any common variable with $v_{i_1},v_{i_2},...,v_{i_k}$ , the polynomial Q misses at least one variable form $v_{i_1},v_{i_2},...,v_{i_k}$.
\end{question}
Let C be the set of points where the variables
in $\{ x_1,x_2,...,x_n \} \bigcup V $are fixed and the variables in U are allowed to take all possible
combination of values. Then
\begin{multline}\sum_{C} F_i(x_1,x_2,...,x_n,v_1,v_2,...,v_m) \\= \sum_{C}v_{i_1},v_{i_2},...,v_{i_k}P(x_1,x_2,...,x_n,V) + \\\sum_{C}Q(x_1,x_2,...,x_n,v_1,v_2,...,v_m)\end{multline}
 \begin{theorm}
 	For any polynomial $F_i$ and subset U,\\ $$\sum_C P(\cdot) = P(\cdot) mod 2$$
 \end{theorm}
\textbf{Proof:} \\
For the polynomial Q is such that none of the terms in Q have the monomial
$v_{i_1},v_{i_2},...,v_{i_k}$ as a factor, summing each of terms in Q over all the $2^k$ possible vectors,the sum value is 0, hence $$\sum_{C}Q=0$$ on the other hand, the coefficient of $P (\cdot)$ in the summation is 1 for only one case $v_{i_1}=v_{i_2}=v_{i_k}=1$ . \\
According to Theorem,
\begin{multline}\sum_{C} F_i(x_1,x_2,...,x_n,v_1,v_2,...,v_m) \\= \sum_{C}v_{i_1},v_{i_2},...,v_{i_k}P(x_1,x_2,...,x_n,V) + \\\sum_{C}Q(x_1,x_2,...,x_n,v_1,v_2,...,v_m)\end{multline}
\cite{wang2013improved}
\subsection{Cube attack on Grain-V1}
The main goal of Cube Attack is to find sufficiently U , then we may be able to
obtain linearity polynomials $P(\cdot)$ . Through query to the cipher obtain the value of
$P(\cdot)$ , and with sufficient number of linearly independent relations in the key bits,
the attacker can easily recover the key K via Gaussian elimination. Cube Attack is
split into two stages:
\subsubsection{The Preprocessing Stage}
In order to find $U$ , we use a linearity check approach, the $IV$ s are formed by
allowing variables in $U$ to take all possible combinations of values while keeping
variables in$ V  = IV/U$ fixed to 0. After the selection of U , then we take 100 random
pairs of keys $(X,Y)$  and check whether 
\begin{multline} \label{eq:cube2}
	\sum_{IV \in C} F_i(X+Y,IV) = \sum_{IV \in C} F_i(X,IV)\\ +\sum_{IV \in C} F_i(Y,IV) + \sum_{IV \in C} F_i(0,IV)
\end{multline}
If [\ref{eq:cube2}] is satisfied for all the 100 random pairs, then the polynomial $P(\cdot)$ is assumed to
be linear in key bits 0. The randomized algorithm presented in 0 to find U starting
with randomly chosen $P(\geq)$  variables and use a linearity test to check whether $P(\cdot)$
is linear. If |U| is too small, then $P(\cdot)$ is likely to be a nonlinear polynomial in the
secret variables. In this case, the attacker adds a public variable to U and checks
again. If |U| is too large, then $P(\cdot)$ will be a constant. And in this case, the attacker
drops one of the public variables form U and checks again. The first problem of this
algorithm is that not all U are tested, for it is chosen randomly and starting from a
random set of variables. The chances of getting U are not expected to be high. So we
select U by adding IV variables one by one. This process can generate all the U
which could produce linear relations $P(\cdot)$ .
\begin{figure}[h]
	\centering
	\includegraphics[width=200px]{cube.png}
	\caption{The Apporch of selecting IV}
\end{figure}

\\[6mm]
You can found the cube founding algoritham [\ref{algo:cube}] following.
\begin{algorithm}[h]
	\KwIn{Set of IV bits $V = {v_0 , . . . , v_{95} }$, cube size c.}
	\KwOut{A good cube $C_I \subset V $of size c}	
	\nl \bf $C_I = \emptyset$;\\
	\nl \bf for ${i=1}$ to ${i=c}$ do\\
	\nl \bf \quad for ${j=0}$ to ${j=95}$ do \\
	\nl \bf \quad \quad $maxround = 0$ ; if $v_j \notin C_I$ then \\
	\nl \bf \quad \quad \quad $\tilde{\mathbf{C_I}}$ =  $C_I \cup v_j$  ; \\
	\nl \bf \quad \quad \quad R = round up to which cube sum for cube variables $\tilde{\mathbf{C_I}}$ is zero;\\
	\nl \bf \quad \quad \quad if $R \geq maxround$ then \\
	\nl \bf \quad \quad \quad \quad $temp = v_j$ ; \\
	\nl \bf \quad \quad \quad \quad $maxround = R$; \\
	\nl \bf \quad \quad \quad end \\
	\nl \bf \quad \quad \quad $C_I$ = $C_I \cup \{temp\}$\\
	\nl \bf \quad \quad end \\
	\nl \bf \quad end \\
	\nl \bf end \\
	\nl \bf return $C_I$; \\\
	
	\caption{Cube Finding Algoritham \cite{1930-5346_2019_0_107}}
	\label{algo:cube}
	
\end{algorithm}
